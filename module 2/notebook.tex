
% Default to the notebook output style

    


% Inherit from the specified cell style.




    
\documentclass[11pt]{article}

    
    
    \usepackage[T1]{fontenc}
    % Nicer default font (+ math font) than Computer Modern for most use cases
    \usepackage{mathpazo}

    % Basic figure setup, for now with no caption control since it's done
    % automatically by Pandoc (which extracts ![](path) syntax from Markdown).
    \usepackage{graphicx}
    % We will generate all images so they have a width \maxwidth. This means
    % that they will get their normal width if they fit onto the page, but
    % are scaled down if they would overflow the margins.
    \makeatletter
    \def\maxwidth{\ifdim\Gin@nat@width>\linewidth\linewidth
    \else\Gin@nat@width\fi}
    \makeatother
    \let\Oldincludegraphics\includegraphics
    % Set max figure width to be 80% of text width, for now hardcoded.
    \renewcommand{\includegraphics}[1]{\Oldincludegraphics[width=.8\maxwidth]{#1}}
    % Ensure that by default, figures have no caption (until we provide a
    % proper Figure object with a Caption API and a way to capture that
    % in the conversion process - todo).
    \usepackage{caption}
    \DeclareCaptionLabelFormat{nolabel}{}
    \captionsetup{labelformat=nolabel}

    \usepackage{adjustbox} % Used to constrain images to a maximum size 
    \usepackage{xcolor} % Allow colors to be defined
    \usepackage{enumerate} % Needed for markdown enumerations to work
    \usepackage{geometry} % Used to adjust the document margins
    \usepackage{amsmath} % Equations
    \usepackage{amssymb} % Equations
    \usepackage{textcomp} % defines textquotesingle
    % Hack from http://tex.stackexchange.com/a/47451/13684:
    \AtBeginDocument{%
        \def\PYZsq{\textquotesingle}% Upright quotes in Pygmentized code
    }
    \usepackage{upquote} % Upright quotes for verbatim code
    \usepackage{eurosym} % defines \euro
    \usepackage[mathletters]{ucs} % Extended unicode (utf-8) support
    \usepackage[utf8x]{inputenc} % Allow utf-8 characters in the tex document
    \usepackage{fancyvrb} % verbatim replacement that allows latex
    \usepackage{grffile} % extends the file name processing of package graphics 
                         % to support a larger range 
    % The hyperref package gives us a pdf with properly built
    % internal navigation ('pdf bookmarks' for the table of contents,
    % internal cross-reference links, web links for URLs, etc.)
    \usepackage{hyperref}
    \usepackage{longtable} % longtable support required by pandoc >1.10
    \usepackage{booktabs}  % table support for pandoc > 1.12.2
    \usepackage[inline]{enumitem} % IRkernel/repr support (it uses the enumerate* environment)
    \usepackage[normalem]{ulem} % ulem is needed to support strikethroughs (\sout)
                                % normalem makes italics be italics, not underlines
    

    
    
    % Colors for the hyperref package
    \definecolor{urlcolor}{rgb}{0,.145,.698}
    \definecolor{linkcolor}{rgb}{.71,0.21,0.01}
    \definecolor{citecolor}{rgb}{.12,.54,.11}

    % ANSI colors
    \definecolor{ansi-black}{HTML}{3E424D}
    \definecolor{ansi-black-intense}{HTML}{282C36}
    \definecolor{ansi-red}{HTML}{E75C58}
    \definecolor{ansi-red-intense}{HTML}{B22B31}
    \definecolor{ansi-green}{HTML}{00A250}
    \definecolor{ansi-green-intense}{HTML}{007427}
    \definecolor{ansi-yellow}{HTML}{DDB62B}
    \definecolor{ansi-yellow-intense}{HTML}{B27D12}
    \definecolor{ansi-blue}{HTML}{208FFB}
    \definecolor{ansi-blue-intense}{HTML}{0065CA}
    \definecolor{ansi-magenta}{HTML}{D160C4}
    \definecolor{ansi-magenta-intense}{HTML}{A03196}
    \definecolor{ansi-cyan}{HTML}{60C6C8}
    \definecolor{ansi-cyan-intense}{HTML}{258F8F}
    \definecolor{ansi-white}{HTML}{C5C1B4}
    \definecolor{ansi-white-intense}{HTML}{A1A6B2}

    % commands and environments needed by pandoc snippets
    % extracted from the output of `pandoc -s`
    \providecommand{\tightlist}{%
      \setlength{\itemsep}{0pt}\setlength{\parskip}{0pt}}
    \DefineVerbatimEnvironment{Highlighting}{Verbatim}{commandchars=\\\{\}}
    % Add ',fontsize=\small' for more characters per line
    \newenvironment{Shaded}{}{}
    \newcommand{\KeywordTok}[1]{\textcolor[rgb]{0.00,0.44,0.13}{\textbf{{#1}}}}
    \newcommand{\DataTypeTok}[1]{\textcolor[rgb]{0.56,0.13,0.00}{{#1}}}
    \newcommand{\DecValTok}[1]{\textcolor[rgb]{0.25,0.63,0.44}{{#1}}}
    \newcommand{\BaseNTok}[1]{\textcolor[rgb]{0.25,0.63,0.44}{{#1}}}
    \newcommand{\FloatTok}[1]{\textcolor[rgb]{0.25,0.63,0.44}{{#1}}}
    \newcommand{\CharTok}[1]{\textcolor[rgb]{0.25,0.44,0.63}{{#1}}}
    \newcommand{\StringTok}[1]{\textcolor[rgb]{0.25,0.44,0.63}{{#1}}}
    \newcommand{\CommentTok}[1]{\textcolor[rgb]{0.38,0.63,0.69}{\textit{{#1}}}}
    \newcommand{\OtherTok}[1]{\textcolor[rgb]{0.00,0.44,0.13}{{#1}}}
    \newcommand{\AlertTok}[1]{\textcolor[rgb]{1.00,0.00,0.00}{\textbf{{#1}}}}
    \newcommand{\FunctionTok}[1]{\textcolor[rgb]{0.02,0.16,0.49}{{#1}}}
    \newcommand{\RegionMarkerTok}[1]{{#1}}
    \newcommand{\ErrorTok}[1]{\textcolor[rgb]{1.00,0.00,0.00}{\textbf{{#1}}}}
    \newcommand{\NormalTok}[1]{{#1}}
    
    % Additional commands for more recent versions of Pandoc
    \newcommand{\ConstantTok}[1]{\textcolor[rgb]{0.53,0.00,0.00}{{#1}}}
    \newcommand{\SpecialCharTok}[1]{\textcolor[rgb]{0.25,0.44,0.63}{{#1}}}
    \newcommand{\VerbatimStringTok}[1]{\textcolor[rgb]{0.25,0.44,0.63}{{#1}}}
    \newcommand{\SpecialStringTok}[1]{\textcolor[rgb]{0.73,0.40,0.53}{{#1}}}
    \newcommand{\ImportTok}[1]{{#1}}
    \newcommand{\DocumentationTok}[1]{\textcolor[rgb]{0.73,0.13,0.13}{\textit{{#1}}}}
    \newcommand{\AnnotationTok}[1]{\textcolor[rgb]{0.38,0.63,0.69}{\textbf{\textit{{#1}}}}}
    \newcommand{\CommentVarTok}[1]{\textcolor[rgb]{0.38,0.63,0.69}{\textbf{\textit{{#1}}}}}
    \newcommand{\VariableTok}[1]{\textcolor[rgb]{0.10,0.09,0.49}{{#1}}}
    \newcommand{\ControlFlowTok}[1]{\textcolor[rgb]{0.00,0.44,0.13}{\textbf{{#1}}}}
    \newcommand{\OperatorTok}[1]{\textcolor[rgb]{0.40,0.40,0.40}{{#1}}}
    \newcommand{\BuiltInTok}[1]{{#1}}
    \newcommand{\ExtensionTok}[1]{{#1}}
    \newcommand{\PreprocessorTok}[1]{\textcolor[rgb]{0.74,0.48,0.00}{{#1}}}
    \newcommand{\AttributeTok}[1]{\textcolor[rgb]{0.49,0.56,0.16}{{#1}}}
    \newcommand{\InformationTok}[1]{\textcolor[rgb]{0.38,0.63,0.69}{\textbf{\textit{{#1}}}}}
    \newcommand{\WarningTok}[1]{\textcolor[rgb]{0.38,0.63,0.69}{\textbf{\textit{{#1}}}}}
    
    
    % Define a nice break command that doesn't care if a line doesn't already
    % exist.
    \def\br{\hspace*{\fill} \\* }
    % Math Jax compatability definitions
    \def\gt{>}
    \def\lt{<}
    % Document parameters
    \title{Module Two Discussion}
    
    
    

    % Pygments definitions
    
\makeatletter
\def\PY@reset{\let\PY@it=\relax \let\PY@bf=\relax%
    \let\PY@ul=\relax \let\PY@tc=\relax%
    \let\PY@bc=\relax \let\PY@ff=\relax}
\def\PY@tok#1{\csname PY@tok@#1\endcsname}
\def\PY@toks#1+{\ifx\relax#1\empty\else%
    \PY@tok{#1}\expandafter\PY@toks\fi}
\def\PY@do#1{\PY@bc{\PY@tc{\PY@ul{%
    \PY@it{\PY@bf{\PY@ff{#1}}}}}}}
\def\PY#1#2{\PY@reset\PY@toks#1+\relax+\PY@do{#2}}

\expandafter\def\csname PY@tok@w\endcsname{\def\PY@tc##1{\textcolor[rgb]{0.73,0.73,0.73}{##1}}}
\expandafter\def\csname PY@tok@c\endcsname{\let\PY@it=\textit\def\PY@tc##1{\textcolor[rgb]{0.25,0.50,0.50}{##1}}}
\expandafter\def\csname PY@tok@cp\endcsname{\def\PY@tc##1{\textcolor[rgb]{0.74,0.48,0.00}{##1}}}
\expandafter\def\csname PY@tok@k\endcsname{\let\PY@bf=\textbf\def\PY@tc##1{\textcolor[rgb]{0.00,0.50,0.00}{##1}}}
\expandafter\def\csname PY@tok@kp\endcsname{\def\PY@tc##1{\textcolor[rgb]{0.00,0.50,0.00}{##1}}}
\expandafter\def\csname PY@tok@kt\endcsname{\def\PY@tc##1{\textcolor[rgb]{0.69,0.00,0.25}{##1}}}
\expandafter\def\csname PY@tok@o\endcsname{\def\PY@tc##1{\textcolor[rgb]{0.40,0.40,0.40}{##1}}}
\expandafter\def\csname PY@tok@ow\endcsname{\let\PY@bf=\textbf\def\PY@tc##1{\textcolor[rgb]{0.67,0.13,1.00}{##1}}}
\expandafter\def\csname PY@tok@nb\endcsname{\def\PY@tc##1{\textcolor[rgb]{0.00,0.50,0.00}{##1}}}
\expandafter\def\csname PY@tok@nf\endcsname{\def\PY@tc##1{\textcolor[rgb]{0.00,0.00,1.00}{##1}}}
\expandafter\def\csname PY@tok@nc\endcsname{\let\PY@bf=\textbf\def\PY@tc##1{\textcolor[rgb]{0.00,0.00,1.00}{##1}}}
\expandafter\def\csname PY@tok@nn\endcsname{\let\PY@bf=\textbf\def\PY@tc##1{\textcolor[rgb]{0.00,0.00,1.00}{##1}}}
\expandafter\def\csname PY@tok@ne\endcsname{\let\PY@bf=\textbf\def\PY@tc##1{\textcolor[rgb]{0.82,0.25,0.23}{##1}}}
\expandafter\def\csname PY@tok@nv\endcsname{\def\PY@tc##1{\textcolor[rgb]{0.10,0.09,0.49}{##1}}}
\expandafter\def\csname PY@tok@no\endcsname{\def\PY@tc##1{\textcolor[rgb]{0.53,0.00,0.00}{##1}}}
\expandafter\def\csname PY@tok@nl\endcsname{\def\PY@tc##1{\textcolor[rgb]{0.63,0.63,0.00}{##1}}}
\expandafter\def\csname PY@tok@ni\endcsname{\let\PY@bf=\textbf\def\PY@tc##1{\textcolor[rgb]{0.60,0.60,0.60}{##1}}}
\expandafter\def\csname PY@tok@na\endcsname{\def\PY@tc##1{\textcolor[rgb]{0.49,0.56,0.16}{##1}}}
\expandafter\def\csname PY@tok@nt\endcsname{\let\PY@bf=\textbf\def\PY@tc##1{\textcolor[rgb]{0.00,0.50,0.00}{##1}}}
\expandafter\def\csname PY@tok@nd\endcsname{\def\PY@tc##1{\textcolor[rgb]{0.67,0.13,1.00}{##1}}}
\expandafter\def\csname PY@tok@s\endcsname{\def\PY@tc##1{\textcolor[rgb]{0.73,0.13,0.13}{##1}}}
\expandafter\def\csname PY@tok@sd\endcsname{\let\PY@it=\textit\def\PY@tc##1{\textcolor[rgb]{0.73,0.13,0.13}{##1}}}
\expandafter\def\csname PY@tok@si\endcsname{\let\PY@bf=\textbf\def\PY@tc##1{\textcolor[rgb]{0.73,0.40,0.53}{##1}}}
\expandafter\def\csname PY@tok@se\endcsname{\let\PY@bf=\textbf\def\PY@tc##1{\textcolor[rgb]{0.73,0.40,0.13}{##1}}}
\expandafter\def\csname PY@tok@sr\endcsname{\def\PY@tc##1{\textcolor[rgb]{0.73,0.40,0.53}{##1}}}
\expandafter\def\csname PY@tok@ss\endcsname{\def\PY@tc##1{\textcolor[rgb]{0.10,0.09,0.49}{##1}}}
\expandafter\def\csname PY@tok@sx\endcsname{\def\PY@tc##1{\textcolor[rgb]{0.00,0.50,0.00}{##1}}}
\expandafter\def\csname PY@tok@m\endcsname{\def\PY@tc##1{\textcolor[rgb]{0.40,0.40,0.40}{##1}}}
\expandafter\def\csname PY@tok@gh\endcsname{\let\PY@bf=\textbf\def\PY@tc##1{\textcolor[rgb]{0.00,0.00,0.50}{##1}}}
\expandafter\def\csname PY@tok@gu\endcsname{\let\PY@bf=\textbf\def\PY@tc##1{\textcolor[rgb]{0.50,0.00,0.50}{##1}}}
\expandafter\def\csname PY@tok@gd\endcsname{\def\PY@tc##1{\textcolor[rgb]{0.63,0.00,0.00}{##1}}}
\expandafter\def\csname PY@tok@gi\endcsname{\def\PY@tc##1{\textcolor[rgb]{0.00,0.63,0.00}{##1}}}
\expandafter\def\csname PY@tok@gr\endcsname{\def\PY@tc##1{\textcolor[rgb]{1.00,0.00,0.00}{##1}}}
\expandafter\def\csname PY@tok@ge\endcsname{\let\PY@it=\textit}
\expandafter\def\csname PY@tok@gs\endcsname{\let\PY@bf=\textbf}
\expandafter\def\csname PY@tok@gp\endcsname{\let\PY@bf=\textbf\def\PY@tc##1{\textcolor[rgb]{0.00,0.00,0.50}{##1}}}
\expandafter\def\csname PY@tok@go\endcsname{\def\PY@tc##1{\textcolor[rgb]{0.53,0.53,0.53}{##1}}}
\expandafter\def\csname PY@tok@gt\endcsname{\def\PY@tc##1{\textcolor[rgb]{0.00,0.27,0.87}{##1}}}
\expandafter\def\csname PY@tok@err\endcsname{\def\PY@bc##1{\setlength{\fboxsep}{0pt}\fcolorbox[rgb]{1.00,0.00,0.00}{1,1,1}{\strut ##1}}}
\expandafter\def\csname PY@tok@kc\endcsname{\let\PY@bf=\textbf\def\PY@tc##1{\textcolor[rgb]{0.00,0.50,0.00}{##1}}}
\expandafter\def\csname PY@tok@kd\endcsname{\let\PY@bf=\textbf\def\PY@tc##1{\textcolor[rgb]{0.00,0.50,0.00}{##1}}}
\expandafter\def\csname PY@tok@kn\endcsname{\let\PY@bf=\textbf\def\PY@tc##1{\textcolor[rgb]{0.00,0.50,0.00}{##1}}}
\expandafter\def\csname PY@tok@kr\endcsname{\let\PY@bf=\textbf\def\PY@tc##1{\textcolor[rgb]{0.00,0.50,0.00}{##1}}}
\expandafter\def\csname PY@tok@bp\endcsname{\def\PY@tc##1{\textcolor[rgb]{0.00,0.50,0.00}{##1}}}
\expandafter\def\csname PY@tok@fm\endcsname{\def\PY@tc##1{\textcolor[rgb]{0.00,0.00,1.00}{##1}}}
\expandafter\def\csname PY@tok@vc\endcsname{\def\PY@tc##1{\textcolor[rgb]{0.10,0.09,0.49}{##1}}}
\expandafter\def\csname PY@tok@vg\endcsname{\def\PY@tc##1{\textcolor[rgb]{0.10,0.09,0.49}{##1}}}
\expandafter\def\csname PY@tok@vi\endcsname{\def\PY@tc##1{\textcolor[rgb]{0.10,0.09,0.49}{##1}}}
\expandafter\def\csname PY@tok@vm\endcsname{\def\PY@tc##1{\textcolor[rgb]{0.10,0.09,0.49}{##1}}}
\expandafter\def\csname PY@tok@sa\endcsname{\def\PY@tc##1{\textcolor[rgb]{0.73,0.13,0.13}{##1}}}
\expandafter\def\csname PY@tok@sb\endcsname{\def\PY@tc##1{\textcolor[rgb]{0.73,0.13,0.13}{##1}}}
\expandafter\def\csname PY@tok@sc\endcsname{\def\PY@tc##1{\textcolor[rgb]{0.73,0.13,0.13}{##1}}}
\expandafter\def\csname PY@tok@dl\endcsname{\def\PY@tc##1{\textcolor[rgb]{0.73,0.13,0.13}{##1}}}
\expandafter\def\csname PY@tok@s2\endcsname{\def\PY@tc##1{\textcolor[rgb]{0.73,0.13,0.13}{##1}}}
\expandafter\def\csname PY@tok@sh\endcsname{\def\PY@tc##1{\textcolor[rgb]{0.73,0.13,0.13}{##1}}}
\expandafter\def\csname PY@tok@s1\endcsname{\def\PY@tc##1{\textcolor[rgb]{0.73,0.13,0.13}{##1}}}
\expandafter\def\csname PY@tok@mb\endcsname{\def\PY@tc##1{\textcolor[rgb]{0.40,0.40,0.40}{##1}}}
\expandafter\def\csname PY@tok@mf\endcsname{\def\PY@tc##1{\textcolor[rgb]{0.40,0.40,0.40}{##1}}}
\expandafter\def\csname PY@tok@mh\endcsname{\def\PY@tc##1{\textcolor[rgb]{0.40,0.40,0.40}{##1}}}
\expandafter\def\csname PY@tok@mi\endcsname{\def\PY@tc##1{\textcolor[rgb]{0.40,0.40,0.40}{##1}}}
\expandafter\def\csname PY@tok@il\endcsname{\def\PY@tc##1{\textcolor[rgb]{0.40,0.40,0.40}{##1}}}
\expandafter\def\csname PY@tok@mo\endcsname{\def\PY@tc##1{\textcolor[rgb]{0.40,0.40,0.40}{##1}}}
\expandafter\def\csname PY@tok@ch\endcsname{\let\PY@it=\textit\def\PY@tc##1{\textcolor[rgb]{0.25,0.50,0.50}{##1}}}
\expandafter\def\csname PY@tok@cm\endcsname{\let\PY@it=\textit\def\PY@tc##1{\textcolor[rgb]{0.25,0.50,0.50}{##1}}}
\expandafter\def\csname PY@tok@cpf\endcsname{\let\PY@it=\textit\def\PY@tc##1{\textcolor[rgb]{0.25,0.50,0.50}{##1}}}
\expandafter\def\csname PY@tok@c1\endcsname{\let\PY@it=\textit\def\PY@tc##1{\textcolor[rgb]{0.25,0.50,0.50}{##1}}}
\expandafter\def\csname PY@tok@cs\endcsname{\let\PY@it=\textit\def\PY@tc##1{\textcolor[rgb]{0.25,0.50,0.50}{##1}}}

\def\PYZbs{\char`\\}
\def\PYZus{\char`\_}
\def\PYZob{\char`\{}
\def\PYZcb{\char`\}}
\def\PYZca{\char`\^}
\def\PYZam{\char`\&}
\def\PYZlt{\char`\<}
\def\PYZgt{\char`\>}
\def\PYZsh{\char`\#}
\def\PYZpc{\char`\%}
\def\PYZdl{\char`\$}
\def\PYZhy{\char`\-}
\def\PYZsq{\char`\'}
\def\PYZdq{\char`\"}
\def\PYZti{\char`\~}
% for compatibility with earlier versions
\def\PYZat{@}
\def\PYZlb{[}
\def\PYZrb{]}
\makeatother


    % Exact colors from NB
    \definecolor{incolor}{rgb}{0.0, 0.0, 0.5}
    \definecolor{outcolor}{rgb}{0.545, 0.0, 0.0}



    
    % Prevent overflowing lines due to hard-to-break entities
    \sloppy 
    % Setup hyperref package
    \hypersetup{
      breaklinks=true,  % so long urls are correctly broken across lines
      colorlinks=true,
      urlcolor=urlcolor,
      linkcolor=linkcolor,
      citecolor=citecolor,
      }
    % Slightly bigger margins than the latex defaults
    
    \geometry{verbose,tmargin=1in,bmargin=1in,lmargin=1in,rmargin=1in}
    
    

    \begin{document}
    
    
    \maketitle
    
    

    
    \hypertarget{module-two-discussion}{%
\section{Module Two Discussion}\label{module-two-discussion}}

This notebook contains the step-by-step directions for your Module Two
discussion.

    \hypertarget{initial-post-due-thursday}{%
\subsection{Initial Post (due
Thursday)}\label{initial-post-due-thursday}}

\begin{center}\rule{0.5\linewidth}{\linethickness}\end{center}

    \hypertarget{step-1-generating-population-data}{%
\subsubsection{Step 1: Generating population
data}\label{step-1-generating-population-data}}

This block of python code will generate a unique TPCP population data of
size 500 observations that you will use in module 2 discussion. We will
use the gamma distribution to create this data set with a skewed
distribution. The numpy module in Python allows us to randomly generate
data using the gamma distribution and return it in an array. The data
set will be saved in a Python dataframe that will be used in later
calculations.

    \begin{Verbatim}[commandchars=\\\{\}]
{\color{incolor}In [{\color{incolor}99}]:} \PY{k+kn}{import} \PY{n+nn}{pandas} \PY{k}{as} \PY{n+nn}{pd}
         \PY{k+kn}{import} \PY{n+nn}{matplotlib}\PY{n+nn}{.}\PY{n+nn}{pyplot} \PY{k}{as} \PY{n+nn}{plt}
         \PY{k+kn}{import} \PY{n+nn}{numpy} \PY{k}{as} \PY{n+nn}{np}
         
         \PY{c+c1}{\PYZsh{} using gamma distribution to randomly generate 500 observations. }
         \PY{n}{shape}\PY{p}{,} \PY{n}{scale} \PY{o}{=} \PY{l+m+mf}{1.95}\PY{p}{,} \PY{l+m+mf}{2.5}
         \PY{n}{tpcp} \PY{o}{=} \PY{l+m+mi}{100}\PY{o}{*}\PY{n}{np}\PY{o}{.}\PY{n}{random}\PY{o}{.}\PY{n}{gamma}\PY{p}{(}\PY{n}{shape}\PY{p}{,} \PY{n}{scale}\PY{p}{,} \PY{l+m+mi}{500}\PY{p}{)}
         
         \PY{c+c1}{\PYZsh{} pandas library can be used to convert the array into a dataframe with the column name TPCP.}
         \PY{n}{tpcp\PYZus{}df} \PY{o}{=} \PY{n}{pd}\PY{o}{.}\PY{n}{DataFrame}\PY{p}{(}\PY{n}{tpcp}\PY{p}{,} \PY{n}{columns}\PY{o}{=}\PY{p}{[}\PY{l+s+s1}{\PYZsq{}}\PY{l+s+s1}{TPCP}\PY{l+s+s1}{\PYZsq{}}\PY{p}{]}\PY{p}{)}
         \PY{n}{tpcp\PYZus{}df} \PY{o}{=} \PY{n}{tpcp\PYZus{}df}\PY{o}{.}\PY{n}{round}\PY{p}{(}\PY{l+m+mi}{0}\PY{p}{)}
         
         \PY{c+c1}{\PYZsh{} print the dataframe to see the first 5 and last 5 observations (note that the index of dataframe starts at 0)}
         \PY{n+nb}{print}\PY{p}{(}\PY{l+s+s2}{\PYZdq{}}\PY{l+s+s2}{TPCP data frame}\PY{l+s+se}{\PYZbs{}n}\PY{l+s+s2}{\PYZdq{}}\PY{p}{)}
         \PY{n+nb}{print}\PY{p}{(}\PY{n}{tpcp\PYZus{}df}\PY{p}{)}
\end{Verbatim}


    \begin{Verbatim}[commandchars=\\\{\}]
TPCP data frame

      TPCP
0    477.0
1    368.0
2    252.0
3    196.0
4    723.0
..     {\ldots}
495  612.0
496   86.0
497  542.0
498  322.0
499  843.0

[500 rows x 1 columns]

    \end{Verbatim}

    \hypertarget{step-2-histogram-plot-of-population-data}{%
\subsubsection{Step 2: Histogram plot of population
data}\label{step-2-histogram-plot-of-population-data}}

The matplotlib module in Python can be used to create a histogram plot
of the population data. This plot allows us to visualize the data
distribution and confirm that it is skewed. We will use 50 bins in the
histogram to display the distribution.

    \begin{Verbatim}[commandchars=\\\{\}]
{\color{incolor}In [{\color{incolor}100}]:} \PY{c+c1}{\PYZsh{} create a figure for the plot. }
          \PY{n}{fig}\PY{p}{,} \PY{n}{ax} \PY{o}{=} \PY{n}{plt}\PY{o}{.}\PY{n}{subplots}\PY{p}{(}\PY{p}{)}
          
          \PY{c+c1}{\PYZsh{} a histogram plot with 50 bins of TPCP population data. }
          \PY{n}{plt}\PY{o}{.}\PY{n}{hist}\PY{p}{(}\PY{n}{tpcp\PYZus{}df}\PY{p}{[}\PY{l+s+s1}{\PYZsq{}}\PY{l+s+s1}{TPCP}\PY{l+s+s1}{\PYZsq{}}\PY{p}{]}\PY{p}{,} \PY{n}{bins}\PY{o}{=}\PY{l+m+mi}{50}\PY{p}{)}
          
          \PY{c+c1}{\PYZsh{} setting a title for the plot, x\PYZhy{}axis and y\PYZhy{}axis}
          \PY{n}{plt}\PY{o}{.}\PY{n}{title}\PY{p}{(}\PY{l+s+s1}{\PYZsq{}}\PY{l+s+s1}{TPCP population distribution}\PY{l+s+s1}{\PYZsq{}}\PY{p}{,} \PY{n}{fontsize}\PY{o}{=}\PY{l+m+mi}{20}\PY{p}{)}
          \PY{n}{ax}\PY{o}{.}\PY{n}{set\PYZus{}xlabel}\PY{p}{(}\PY{l+s+s1}{\PYZsq{}}\PY{l+s+s1}{TPCP}\PY{l+s+s1}{\PYZsq{}}\PY{p}{)}
          \PY{n}{ax}\PY{o}{.}\PY{n}{set\PYZus{}ylabel}\PY{p}{(}\PY{l+s+s1}{\PYZsq{}}\PY{l+s+s1}{Frequency}\PY{l+s+s1}{\PYZsq{}}\PY{p}{)}
          
          \PY{c+c1}{\PYZsh{} show the plot}
          \PY{n}{plt}\PY{o}{.}\PY{n}{show}\PY{p}{(}\PY{p}{)}
\end{Verbatim}


    \begin{center}
    \adjustimage{max size={0.9\linewidth}{0.9\paperheight}}{output_5_0.png}
    \end{center}
    { \hspace*{\fill} \\}
    
    \hypertarget{step-3-calculating-the-population-mean}{%
\subsubsection{Step 3: Calculating the population
mean}\label{step-3-calculating-the-population-mean}}

We can use the ``mean'' method of a dataframe to calculate the
population mean.

    \begin{Verbatim}[commandchars=\\\{\}]
{\color{incolor}In [{\color{incolor}101}]:} \PY{n}{pop\PYZus{}mean} \PY{o}{=} \PY{n}{tpcp\PYZus{}df}\PY{p}{[}\PY{l+s+s1}{\PYZsq{}}\PY{l+s+s1}{TPCP}\PY{l+s+s1}{\PYZsq{}}\PY{p}{]}\PY{o}{.}\PY{n}{mean}\PY{p}{(}\PY{p}{)}
          \PY{n+nb}{print}\PY{p}{(}\PY{l+s+s2}{\PYZdq{}}\PY{l+s+s2}{Population mean =}\PY{l+s+s2}{\PYZdq{}}\PY{p}{,}\PY{n}{pop\PYZus{}mean}\PY{p}{)}
\end{Verbatim}


    \begin{Verbatim}[commandchars=\\\{\}]
Population mean = 486.056

    \end{Verbatim}

    \hypertarget{step-4-one-random-sample-from-the-population-data}{%
\subsubsection{Step 4: One random sample from the population
data}\label{step-4-one-random-sample-from-the-population-data}}

This block of code randomly selects one sample (with replacement) of
size 50 data points from the population data. The ``sample'' method of
the dataframe can be used to do the selection. We then calculate the
mean of this sample.

    \begin{Verbatim}[commandchars=\\\{\}]
{\color{incolor}In [{\color{incolor}102}]:} \PY{c+c1}{\PYZsh{} using sample method of the dataframe to select a random sample, with replacement, of size 50.}
          \PY{n}{tpcp\PYZus{}sample\PYZus{}df} \PY{o}{=} \PY{n}{tpcp\PYZus{}df}\PY{o}{.}\PY{n}{sample}\PY{p}{(}\PY{l+m+mi}{50}\PY{p}{,} \PY{n}{replace}\PY{o}{=}\PY{k+kc}{True}\PY{p}{)}
          
          \PY{c+c1}{\PYZsh{} the sample mean}
          \PY{n}{sample\PYZus{}mean} \PY{o}{=} \PY{n}{tpcp\PYZus{}sample\PYZus{}df}\PY{p}{[}\PY{l+s+s1}{\PYZsq{}}\PY{l+s+s1}{TPCP}\PY{l+s+s1}{\PYZsq{}}\PY{p}{]}\PY{o}{.}\PY{n}{mean}\PY{p}{(}\PY{p}{)}
          \PY{n+nb}{print}\PY{p}{(}\PY{l+s+s2}{\PYZdq{}}\PY{l+s+s2}{Sample mean =}\PY{l+s+s2}{\PYZdq{}}\PY{p}{,}\PY{n}{sample\PYZus{}mean}\PY{p}{)}
\end{Verbatim}


    \begin{Verbatim}[commandchars=\\\{\}]
Sample mean = 496.22

    \end{Verbatim}

    \hypertarget{step-5-selecting-1000-samples-and-saving-the-sample-means}{%
\subsubsection{Step 5: Selecting 1000 samples and saving the sample
means}\label{step-5-selecting-1000-samples-and-saving-the-sample-means}}

We will now repeat step 4 1000 times to select 1000 random samples, with
replacement, of size 50 from the population data. We will save the
sample mean of each sample in a Python dataframe.

    \begin{Verbatim}[commandchars=\\\{\}]
{\color{incolor}In [{\color{incolor}103}]:} \PY{c+c1}{\PYZsh{} A for loop will be used to repeat step 4 a thousand times. A python list called means\PYZus{}list can be used to save each mean}
          \PY{n}{means\PYZus{}list} \PY{o}{=} \PY{p}{[}\PY{p}{]}
          \PY{k}{for} \PY{n}{i} \PY{o+ow}{in} \PY{n+nb}{range}\PY{p}{(}\PY{l+m+mi}{1000}\PY{p}{)}\PY{p}{:}
              \PY{n}{tpcp\PYZus{}sample\PYZus{}df} \PY{o}{=} \PY{n}{tpcp\PYZus{}df}\PY{o}{.}\PY{n}{sample}\PY{p}{(}\PY{l+m+mi}{50}\PY{p}{,} \PY{n}{replace}\PY{o}{=}\PY{k+kc}{True}\PY{p}{)}
              \PY{n}{sample\PYZus{}mean} \PY{o}{=} \PY{n}{tpcp\PYZus{}sample\PYZus{}df}\PY{p}{[}\PY{l+s+s1}{\PYZsq{}}\PY{l+s+s1}{TPCP}\PY{l+s+s1}{\PYZsq{}}\PY{p}{]}\PY{o}{.}\PY{n}{mean}\PY{p}{(}\PY{p}{)}
              \PY{n}{means\PYZus{}list}\PY{o}{.}\PY{n}{append}\PY{p}{(}\PY{n}{sample\PYZus{}mean}\PY{p}{)}
              
          \PY{c+c1}{\PYZsh{} creating a python dataframe of means}
          \PY{n}{means\PYZus{}df} \PY{o}{=} \PY{n}{pd}\PY{o}{.}\PY{n}{DataFrame}\PY{p}{(}\PY{n}{means\PYZus{}list}\PY{p}{,} \PY{n}{columns}\PY{o}{=}\PY{p}{[}\PY{l+s+s1}{\PYZsq{}}\PY{l+s+s1}{means}\PY{l+s+s1}{\PYZsq{}}\PY{p}{]}\PY{p}{)}
          \PY{n+nb}{print}\PY{p}{(}\PY{l+s+s2}{\PYZdq{}}\PY{l+s+s2}{Dataframe of 1000 means}\PY{l+s+se}{\PYZbs{}n}\PY{l+s+s2}{\PYZdq{}}\PY{p}{)}
          \PY{n+nb}{print}\PY{p}{(}\PY{n}{means\PYZus{}df}\PY{p}{)}
\end{Verbatim}


    \begin{Verbatim}[commandchars=\\\{\}]
Dataframe of 1000 means

      means
0    501.46
1    514.54
2    466.74
3    502.02
4    433.88
..      {\ldots}
995  450.80
996  510.32
997  540.82
998  415.60
999  508.92

[1000 rows x 1 columns]

    \end{Verbatim}

    \hypertarget{step-6-histogram-plot-of-sample-means-from-step-5}{%
\subsubsection{Step 6: Histogram plot of sample means from step
5}\label{step-6-histogram-plot-of-sample-means-from-step-5}}

Now let us plot the data distribution of the 1000 means from step 5 to
confirm that it approximates a normal (bell-shaped) distribution. Again,
the matplotlib module in Python can be used to create a histogram plot.

    \begin{Verbatim}[commandchars=\\\{\}]
{\color{incolor}In [{\color{incolor}104}]:} \PY{c+c1}{\PYZsh{} create a figure for the plot. }
          \PY{n}{fig}\PY{p}{,} \PY{n}{ax} \PY{o}{=} \PY{n}{plt}\PY{o}{.}\PY{n}{subplots}\PY{p}{(}\PY{p}{)}
          
          \PY{c+c1}{\PYZsh{} a histogram plot with 50 bins of 1000 means. }
          \PY{n}{plt}\PY{o}{.}\PY{n}{hist}\PY{p}{(}\PY{n}{means\PYZus{}df}\PY{p}{[}\PY{l+s+s1}{\PYZsq{}}\PY{l+s+s1}{means}\PY{l+s+s1}{\PYZsq{}}\PY{p}{]}\PY{p}{,} \PY{n}{bins}\PY{o}{=}\PY{l+m+mi}{50}\PY{p}{)}
          
          \PY{c+c1}{\PYZsh{} setting a title for the plot, x\PYZhy{}axis and y\PYZhy{}axis}
          \PY{n}{plt}\PY{o}{.}\PY{n}{title}\PY{p}{(}\PY{l+s+s1}{\PYZsq{}}\PY{l+s+s1}{Distribution of 1000 sample means}\PY{l+s+s1}{\PYZsq{}}\PY{p}{,} \PY{n}{fontsize}\PY{o}{=}\PY{l+m+mi}{20}\PY{p}{)} \PY{c+c1}{\PYZsh{} title}
          \PY{n}{ax}\PY{o}{.}\PY{n}{set\PYZus{}xlabel}\PY{p}{(}\PY{l+s+s1}{\PYZsq{}}\PY{l+s+s1}{Means}\PY{l+s+s1}{\PYZsq{}}\PY{p}{)}
          \PY{n}{ax}\PY{o}{.}\PY{n}{set\PYZus{}ylabel}\PY{p}{(}\PY{l+s+s1}{\PYZsq{}}\PY{l+s+s1}{Frequency}\PY{l+s+s1}{\PYZsq{}}\PY{p}{)}
          
          \PY{c+c1}{\PYZsh{} show the plot}
          \PY{n}{plt}\PY{o}{.}\PY{n}{show}\PY{p}{(}\PY{p}{)}
\end{Verbatim}


    \begin{center}
    \adjustimage{max size={0.9\linewidth}{0.9\paperheight}}{output_13_0.png}
    \end{center}
    { \hspace*{\fill} \\}
    
    \hypertarget{step-7-mean-and-the-standard-deviation-of-the-sample-mean-distribution}{%
\subsubsection{Step 7: Mean and the Standard Deviation of the sample
mean
distribution}\label{step-7-mean-and-the-standard-deviation-of-the-sample-mean-distribution}}

Now let's calculate the ``grand'' mean (grand because it is the mean of
1000 means) and the standard deviation of the distribution of 1000
sample means. From step 6, we can see that this distribution is
approximately normal. Therefore, calculating the mean and the standard
deviation of this distribution will allow us to calculate probabilities
and critical values.

    \begin{Verbatim}[commandchars=\\\{\}]
{\color{incolor}In [{\color{incolor}105}]:} \PY{n}{mean1000} \PY{o}{=} \PY{n}{means\PYZus{}df}\PY{p}{[}\PY{l+s+s1}{\PYZsq{}}\PY{l+s+s1}{means}\PY{l+s+s1}{\PYZsq{}}\PY{p}{]}\PY{o}{.}\PY{n}{mean}\PY{p}{(}\PY{p}{)}
          \PY{n}{std1000} \PY{o}{=} \PY{n}{means\PYZus{}df}\PY{p}{[}\PY{l+s+s1}{\PYZsq{}}\PY{l+s+s1}{means}\PY{l+s+s1}{\PYZsq{}}\PY{p}{]}\PY{o}{.}\PY{n}{std}\PY{p}{(}\PY{p}{)}
          \PY{n+nb}{print}\PY{p}{(}\PY{l+s+s2}{\PYZdq{}}\PY{l+s+s2}{Grand mean (mean of 1000 sample means) =}\PY{l+s+s2}{\PYZdq{}}\PY{p}{,}\PY{n}{mean1000}\PY{p}{)}
          \PY{n+nb}{print}\PY{p}{(}\PY{l+s+s2}{\PYZdq{}}\PY{l+s+s2}{Std Deviation of 1000 sample means =}\PY{l+s+s2}{\PYZdq{}}\PY{p}{,}\PY{n}{std1000}\PY{p}{)}
\end{Verbatim}


    \begin{Verbatim}[commandchars=\\\{\}]
Grand mean (mean of 1000 sample means) = 485.30204
Std Deviation of 1000 sample means = 47.31777427987501

    \end{Verbatim}

    \hypertarget{response-posts-due-sunday}{%
\subsection{Response Posts (due
Sunday)}\label{response-posts-due-sunday}}

\begin{center}\rule{0.5\linewidth}{\linethickness}\end{center}

    \hypertarget{step-8-calculating-probability-of-sample-mean-being-less-than-370}{%
\subsubsection{Step 8: Calculating probability of sample mean being less
than
370}\label{step-8-calculating-probability-of-sample-mean-being-less-than-370}}

The distribution of the sample means (1000 sample means) is
approximately normally distributed (step 6) with the mean and standard
deviation that you calculated in step 7. The submodule scipy.stats
includes support for the normal distribution. This can be used to
calculate probabilities and sample mean values for various scenarios. To
calculate the probability that the sample mean is \textbf{less} than
370, the cdf (cumulative distribution function) method in
scipy.stats.norm can be used.

    \begin{Verbatim}[commandchars=\\\{\}]
{\color{incolor}In [{\color{incolor}106}]:} \PY{k+kn}{import} \PY{n+nn}{scipy}\PY{n+nn}{.}\PY{n+nn}{stats} \PY{k}{as} \PY{n+nn}{st}
          
          \PY{c+c1}{\PYZsh{} Prob(sample mean \PYZlt{} 370) = ?  using cdf (cumulative distribution function) method in scipy.stats.norm}
          \PY{n}{prob\PYZus{}less\PYZus{}370} \PY{o}{=} \PY{n}{st}\PY{o}{.}\PY{n}{norm}\PY{o}{.}\PY{n}{cdf}\PY{p}{(}\PY{l+m+mi}{370}\PY{p}{,} \PY{n}{mean1000}\PY{p}{,} \PY{n}{std1000}\PY{p}{)}
          \PY{n+nb}{print}\PY{p}{(}\PY{l+s+s2}{\PYZdq{}}\PY{l+s+s2}{Prob(sample mean \PYZlt{} 370) =}\PY{l+s+s2}{\PYZdq{}}\PY{p}{,}\PY{n}{prob\PYZus{}less\PYZus{}370}\PY{p}{)}
\end{Verbatim}


    \begin{Verbatim}[commandchars=\\\{\}]
Prob(sample mean < 370) = 0.0074097642568580055

    \end{Verbatim}

    \hypertarget{step-9-calculating-probability-of-sample-mean-being-greater-than-510}{%
\subsubsection{Step 9: Calculating probability of sample mean being
greater than
510}\label{step-9-calculating-probability-of-sample-mean-being-greater-than-510}}

To calculate the probability that the sample mean is \textbf{greater}
than 510, the sf (survival function or simply 1-cdf) method in
scipy.stats.norm can be used.

    \begin{Verbatim}[commandchars=\\\{\}]
{\color{incolor}In [{\color{incolor}107}]:} \PY{c+c1}{\PYZsh{} Prob(sample mean \PYZgt{} 510) = ?   using the sf (survival function or simply 1\PYZhy{}cdf) method in scipy.stats.norm}
          \PY{n}{prob\PYZus{}greater\PYZus{}510} \PY{o}{=} \PY{n}{st}\PY{o}{.}\PY{n}{norm}\PY{o}{.}\PY{n}{sf}\PY{p}{(}\PY{l+m+mi}{510}\PY{p}{,} \PY{n}{mean1000}\PY{p}{,} \PY{n}{std1000}\PY{p}{)}
          \PY{n+nb}{print}\PY{p}{(}\PY{l+s+s2}{\PYZdq{}}\PY{l+s+s2}{Prob(sample mean \PYZgt{} 510) =}\PY{l+s+s2}{\PYZdq{}}\PY{p}{,}\PY{n}{prob\PYZus{}greater\PYZus{}510}\PY{p}{)}
\end{Verbatim}


    \begin{Verbatim}[commandchars=\\\{\}]
Prob(sample mean > 510) = 0.3008492779636046

    \end{Verbatim}

    \hypertarget{step-10-calculating-probability-of-sample-mean-between-495-and-550}{%
\subsubsection{Step 10: Calculating probability of sample mean between
495 and
550}\label{step-10-calculating-probability-of-sample-mean-between-495-and-550}}

We can be creative with the cdf method to calculate the probability that
a sample mean is between 495 and 550.

    \begin{Verbatim}[commandchars=\\\{\}]
{\color{incolor}In [{\color{incolor}108}]:} \PY{c+c1}{\PYZsh{} Prob(495 \PYZlt{} sample mean \PYZlt{} 550) = Prob(sample mean \PYZlt{} 550) \PYZhy{} Prob(sample mean \PYZlt{} 495)}
          \PY{n}{prob\PYZus{}between} \PY{o}{=} \PY{n}{st}\PY{o}{.}\PY{n}{norm}\PY{o}{.}\PY{n}{cdf}\PY{p}{(}\PY{l+m+mi}{550}\PY{p}{,} \PY{n}{mean1000}\PY{p}{,} \PY{n}{std1000}\PY{p}{)} \PY{o}{\PYZhy{}} \PY{n}{st}\PY{o}{.}\PY{n}{norm}\PY{o}{.}\PY{n}{cdf}\PY{p}{(}\PY{l+m+mi}{495}\PY{p}{,} \PY{n}{mean1000}\PY{p}{,} \PY{n}{std1000}\PY{p}{)}
          \PY{n+nb}{print}\PY{p}{(}\PY{l+s+s2}{\PYZdq{}}\PY{l+s+s2}{Prob(495 \PYZlt{} sample mean \PYZlt{} 550) =}\PY{l+s+s2}{\PYZdq{}}\PY{p}{,}\PY{n}{prob\PYZus{}between}\PY{p}{)}
\end{Verbatim}


    \begin{Verbatim}[commandchars=\\\{\}]
Prob(495 < sample mean < 550) = 0.33303966099161264

    \end{Verbatim}

    \hypertarget{step-11-calculating-the-sample-mean-a-if-probability-less-than-a-is-0.15}{%
\subsubsection{Step 11: Calculating the sample mean A if probability
less than A is
0.15}\label{step-11-calculating-the-sample-mean-a-if-probability-less-than-a-is-0.15}}

Now we have to calculate the value of sample mean such that the
probability less than this value is 0.15. We can use the ppf (percent
point function or inverse of cdf) method in scipy.stats.norm to
calculate this value.

    \begin{Verbatim}[commandchars=\\\{\}]
{\color{incolor}In [{\color{incolor}109}]:} \PY{c+c1}{\PYZsh{} Prob(sample mean \PYZlt{} A) = 0.15   using the ppf (percent point function or inverse of cdf) method in scipy.stats.norm}
          \PY{n}{A\PYZus{}sample\PYZus{}mean} \PY{o}{=} \PY{n}{st}\PY{o}{.}\PY{n}{norm}\PY{o}{.}\PY{n}{ppf}\PY{p}{(}\PY{l+m+mf}{0.15}\PY{p}{,} \PY{n}{mean1000}\PY{p}{,} \PY{n}{std1000}\PY{p}{)}
          \PY{n+nb}{print}\PY{p}{(}\PY{l+s+s2}{\PYZdq{}}\PY{l+s+s2}{Prob(sample mean \PYZlt{} A) = 0.15,  A =}\PY{l+s+s2}{\PYZdq{}}\PY{p}{,}\PY{n}{A\PYZus{}sample\PYZus{}mean}\PY{p}{)}
\end{Verbatim}


    \begin{Verbatim}[commandchars=\\\{\}]
Prob(sample mean < A) = 0.15,  A = 436.260318819807

    \end{Verbatim}

    \hypertarget{step-12-calculating-the-sample-mean-b-if-probability-greater-than-b-is-0.01}{%
\subsubsection{Step 12: Calculating the sample mean B if probability
greater than B is
0.01}\label{step-12-calculating-the-sample-mean-b-if-probability-greater-than-b-is-0.01}}

We have to calculate the value of sample mean such that the probability
greater than this value is 0.01. We can use the isf (inverse survival
function) method in scipy.stats.norm to calculate this value.

    \begin{Verbatim}[commandchars=\\\{\}]
{\color{incolor}In [{\color{incolor}110}]:} \PY{c+c1}{\PYZsh{} Prob(sample mean \PYZgt{} B) = 0.01   using the isf (inverse survival function) method in scipy.stats.norm}
          \PY{n}{B\PYZus{}sample\PYZus{}mean} \PY{o}{=} \PY{n}{st}\PY{o}{.}\PY{n}{norm}\PY{o}{.}\PY{n}{isf}\PY{p}{(}\PY{l+m+mf}{0.01}\PY{p}{,} \PY{n}{mean1000}\PY{p}{,} \PY{n}{std1000}\PY{p}{)}
          \PY{n+nb}{print}\PY{p}{(}\PY{l+s+s2}{\PYZdq{}}\PY{l+s+s2}{Prob(sample mean \PYZgt{} B) = 0.01,  B =}\PY{l+s+s2}{\PYZdq{}}\PY{p}{,}\PY{n}{B\PYZus{}sample\PYZus{}mean}\PY{p}{)}
\end{Verbatim}


    \begin{Verbatim}[commandchars=\\\{\}]
Prob(sample mean > B) = 0.01,  B = 595.3796436003315

    \end{Verbatim}


    % Add a bibliography block to the postdoc
    
    
    
    \end{document}
